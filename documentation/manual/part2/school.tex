
\section{School}

In this section we provide an example of the full school year: tuning,
learning and working. Again the dteailed scripts to reproduce the
results can be found in Sec.~\ref{sec:}.

\subsection{Tuning}
Using the data set presented above we tune a Quadratic Discriminant
Analysis (QDA) to a subset of the randomly selected training data.
%
\begin{changemargin}{1.5cm}{1.5cm} 
  \begin{lstlisting}[language=Bash]
    
    hepstore-school -c qda -f data_1.npy data_2.npy -l 0.0 1.0 --only_explore
  \end{lstlisting}
\end{changemargin}
%
This yield a ranked output of parameter choice, see Sec.\ref{sec:}
versus score.
\begin{table}[h!]
  \centering
  \begin{tabular}{c|c}
    parameter & value \\
    \hline \hline
    tol       & $2.64\times10^{-9}$ \\
    reg\_param& $1.18\times10^{-2}$\\
    \hline
    \end{tabular}
  \caption{}
  \label{tab:example_tuning}
\end{table}
%
The best tune parameter values are presented in
Tab.~\ref{tab:example_tuning}.
%
\begin{figure}
  \centering
  \includegraphics[width=0.32\textwidth]{../examples/core/school/cross_validation.pdf}
  \includegraphics[width=0.32\textwidth]{../examples/core/school/learning_curve.pdf}
  \caption{}
  \label{fig:example_tuning}
\end{figure}
%
Using hepstore-plot we present in Fig.~\ref{fig:example_tuning} the
parameter cross validation and the learning curve. We observe stable
and convergent behaviour.

\subsection{Learning}
Using the tuned parameter values from Tab.~\ref{tab:example_tuning} we
train and test the QDA classifier on the full data set.
%
\begin{changemargin}{1.5cm}{1.5cm} 
  \begin{lstlisting}[language=Bash]
    
    hepstore-school -c qda -f data_1.npy data_2.npy -l 0.0 1.0 --reg_param 0.0118 --tol 2.64e-09
  \end{lstlisting}
\end{changemargin}
%
%
\begin{figure}
  \centering
  \includegraphics[width=0.32\textwidth]{../examples/core/school/classifier_output.pdf}
  \includegraphics[width=0.32\textwidth]{../examples/core/school/probability_map.pdf}
  \includegraphics[width=0.32\textwidth]{../examples/core/school/roc.pdf}
  \caption{}
  \label{fig:example_training}
\end{figure}
%
In Fig.~\ref{fig:example_tuning} we display the training and testing
results. From the classifier output distribution we deduce that, as
expected, no over training occured. Furthermore, the probability map
shows that the classifier nicely maps the background versus signal
shape. The ROC curve diplayed may be used as input for statistical
analysis.

\subsection{Working}
In regular learning mode hepstore.core.school automatically dumps the
classifier and the scaler as pickable Python tuple. Therefore, one can
load the classifier and work on unlabeled data. We use the same data
as before. However, this time with a new random seed to produce
independent samples and an unknwon number of signal respectively
background events. Using
%
\begin{changemargin}{1.5cm}{1.5cm} 
  \begin{lstlisting}[language=Bash]
    
    hepstore-school --load qda.pkl -f data_3.npy data_4.npy
  \end{lstlisting}
\end{changemargin}
%
we generate a blinded, e.g. no labels, classifier output distribution.
%
\begin{figure}
  \centering
  \includegraphics[width=0.32\textwidth]{../examples/core/school/blind_distribution.pdf}
  \caption{}
  \label{fig:example_working}
\end{figure}
%
In Fig.~\ref{fig:example_working} we present this distribution as
computed from the new data set without labels.




