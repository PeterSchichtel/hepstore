
\section{Statistics}

In general, one can perform statisticla analyses once the pdf's for
all processes are known. In addition, ROC curves contain all the
information needed to perform certain statistical analyses. In the
following we use the results produced with hepstore-school to compute
statistical quantities.

\subsection{Fit a Signal}
We like to estimate the number of signal and background events
contained in the joined data set data\_3.npy and data\_4.npy. As pdf
we use the trained and labeled classifier output distribution,
computed in Sec.~\ref{sec:}. As data the blinded distribution from
Sec.~\ref{sec:}.
%
\begin{changemargin}{1.5cm}{1.5cm} 
  \begin{lstlisting}[language=Bash]
    
    hepstore-statistics --fit --data blinded.npy --pdf cls_pdf_background.npy cls_pdf_signal.npy
  \end{lstlisting}
\end{changemargin}
%
We find $9974.5$ background events and $503.2$ signal events, whereas
the true value was $10000$ respectively $500$.

\subsection{Upper Bound on Signal Cross Section}
Next let us use the ROC curve from Sec.~\ref{sec:} to compute the
upper limit on a possible signal cross section. 
%
\begin{changemargin}{1.5cm}{1.5cm} 
  \begin{lstlisting}[language=Bash]
    
    hepstore-statistics --limit --roc roc.npy --xsec_b 10.0 --luminosity 100.0
  \end{lstlisting}
\end{changemargin}
%
Assuming a luminosity $\mathcal{L} = 100$ and a background cross
section of $\sigma_B = 10$, we compute an upper limit on the signal
cross section of $\sigma_s \le 0.131$. We use the ROC curve obtained
in the previous section. The optimal working point is at $\epsilon_S =
39\%$, corresponding to $\epsilon_B = 0.0014$.

\subsection{Poissonian Significance}
For given expected signal and background number of events we can
compute the expected Poissonian significance as function of the
classifier output.
%
\begin{changemargin}{1.5cm}{1.5cm} 
  \begin{lstlisting}[language=Bash]
    
    hepstore-statistics --significance --cls_b cls_pdf_background.npy --cls_s cls_pdf_signal.npy --xsec_s 0.5 --xsec_b 10. --luminosity 100.
  \end{lstlisting}
\end{changemargin}
%
%
\begin{figure}
  \centering
  \includegraphics[width=0.32\textwidth]{../examples/core/statistic/significance.pdf}
  \caption{}
  \label{fig:example_statistic}
\end{figure}
%
In Fig.~\ref{fig:example_statistic} we present the classifier output
cut dependent significance. Furthermore, we show the corresponding
signal and background efficiencies, $\epsilon_S$ and $\epsilon_B$.XS

%\subsection{Significance from Loglikelihood}

