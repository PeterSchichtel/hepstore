
In this chapter we shortly introduce each of \heptore s core
modules. The next part of the manual actually contains real life
examples ready for use out of the box. The code snippets presented in
the subsequent sections merely presents the anticipated complexity of
the module in question. Furthermore, the full code is published in the
appendix.

\section{hepstore.core.docker\_interface}
\index{hepstore.core.docker\_interface}

The docker\_interface module is at the heart of \hepstore s
reproducibility program. It is a simple yet elegant piece of
\python~code. The interface allowes to run any command in any
\docker~container without any need to know about the native
\docker~commands themselves. It can be used in \python~via
%
\begin{changemargin}{1.5cm}{1.5cm}
  \centering
  \begin{lstlisting}
    
    # import module
    import hepstore.core.docker_interface

    # user vars
    docker_repository    = 'repository_for_pull'
    docker_image         = 'name_of_image'
    docker_image_version = 'version_of_subimage'
    directory_path       = 'mount_local_dir'
    list_of_arguments    = [
      'commands',
      'piped',
      'to',
      'container',
      ]
    
    # generate the app
    app = hepstore.core.docker_interafce.DockerIF(    
        image     = os.path.join( docker_repository,
                                  docker_image,
        ),
        version   = docker_image_version,
        verbose   = True,
    )

    # run the app
    app.run(
        directory = directory_path,
        args      = list_of_arguments,
    )
  \end{lstlisting}
\end{changemargin}
%
Additionally, we provide a command line tool which can be invoked via
%
\begin{changemargin}{1.5cm}{1.5cm}
  \centering
  \begin{lstlisting}[language=Bash]
    
    hepstore-docker list_of_arguments
  \end{lstlisting}
\end{changemargin}
%
For a list of valid arguments use the '-h' flag.

\section{hepstore.core.plotter}
\index{hepstore.core.plotter}

Another important building block of \hepstore~is its own plotting
module. Via the command line
%
\begin{changemargin}{1.5cm}{1.5cm}
  \centering
  \begin{lstlisting}[language=Bash]
    
    hepstore-plot list_of_arguments
  \end{lstlisting}
\end{changemargin}
%
produces any kind of collection of 1D or 2D figures. The plotting
module allows not only for efficient plublishable figure production,
but comes along as \python~class for usability in user modules
%
\begin{changemargin}{1.5cm}{1.5cm}
  \centering
  \begin{lstlisting}
    
    # import module
    import hepstore.core.plotter

    # create a figure
    figure = hepstore.core.plotter.Figure()
    
  \end{lstlisting}
\end{changemargin}
%
At the time being \hepstore~suppoert the following displays
%
\begin{enumerate}
\item scatter
\item histogram
\item errorbar
\item line
\item errorband
\item contour
\end{enumerate}
%


\section{hepstore.core.school}
One of the themes very interesting to particle physiscts and also
recieving more and more attention in the phenomenolgical community is
machine learning. 'hepstore.school' is an attempt to provide a genric
interface to python's sklearn package. our school consists of a
teacher, a student and, of course, a book. the teacher advices the
student which algorithm to use, where upon the student takes the
algorithm from the book and is able to explore and train itself to
learn from the data provided.

\subsubsection{Classifiers}

lda,qcd,svc,mlp

\subsubsection{Tuning}

As it is not a prior clear which parameters to chose for an algorithm,
'hepstore-school' provides the switch '-{}-only\_explore' which will
try to automatically find the best parameters for the given training
set. In addition for all numerical parameters it provides cross
validation plots to check for over respectively under
performance. Note that some classifiers might have a large parameter
space and hence can not be tunde in a fully automated way, yet.

\subsubsection{Training}

Training is performed automattically on a $75\%$ subset of the data provided.

\subsubsection{Testing}

Training is performed automattically on a $25\%$ subset of the data provided.

\section{hepstore.core.statistics}

\section{hepstore.core.tools}

\section{hepstore.core.errors}

\section{hepstore.core.multiprocess}
