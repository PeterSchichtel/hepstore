


In the following we will outly the core modules hepstore is composed
of. We will introduce their usage and motivation and give examples of
their simple usability.
 We show
the source code in the following:
%
\begin{changemargin}{1.5cm}{1.5cm} 
  \lstinputlisting[numbers=left,firstnumber=8,firstline=8]{../../hepstore/core/docker_interface.py}
\end{changemargin}
%
As a starting point using this interface we provide simple command
line tools for
%
\begin{itemize}
\item Herwig 7.0 ~$~~\rightarrow$ 'hepstore-herwig',
\item Sherpa 2.2 ~$~~\rightarrow$ 'hepstore-sherpa',
\item Corsika 7.4 $~~\rightarrow$ 'hepstore-corsika',
\end{itemize}
%
in accordance with the points raised above: frozen state and platform
independent. The actual code and the docker files can be found in
App.~\ref{app:docker}. Note, however, that Sherpa actually has an
official docker account, where we pull their image from. It is also
the aim of this work to motivate further official docker images from
code developers.


\section{hepstore.core.docker\_interface}
\index{hepstore.core.docker\_interface}

\section{hepstore.core.plotter}

In the case where you are not able to use the rivet or madanalysis
plotting capabilities, \hepstore provides a simple user interface
'hepstore-plot' which can plot any one or two dimensional
representation of your data\footnote{provided in .npy
  format}. Currently the following kinds of plots are suported:
scatter, histogram, errorbar, line, errorband, and contour.

\subsubsection{Data Format}

\hepstore~uses the '.npy' format

\section{hepstore.core.school}
One of the themes very interesting to particle physiscts and also
recieving more and more attention in the phenomenolgical community is
machine learning. 'hepstore.school' is an attempt to provide a genric
interface to python's sklearn package. our school consists of a
teacher, a student and, of course, a book. the teacher advices the
student which algorithm to use, where upon the student takes the
algorithm from the book and is able to explore and train itself to
learn from the data provided.

\subsubsection{Classifiers}

lda,qcd,svc,mlp

\subsubsection{Tuning}

As it is not a prior clear which parameters to chose for an algorithm,
'hepstore-school' provides the switch '-{}-only\_explore' which will
try to automatically find the best parameters for the given training
set. In addition for all numerical parameters it provides cross
validation plots to check for over respectively under
performance. Note that some classifiers might have a large parameter
space and hence can not be tunde in a fully automated way, yet.

\subsubsection{Training}

Training is performed automattically on a $75\%$ subset of the data provided.

\subsubsection{Testing}

Training is performed automattically on a $25\%$ subset of the data provided.

\section{hepstore.core.statistics}

\section{hepstore.core.tools}

\section{hepstore.core.errors}

\section{hepstore.core.multiprocess}
