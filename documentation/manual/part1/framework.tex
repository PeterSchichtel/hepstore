
The framework is the part of the code collecting the steering of
reproducible analyses. It consists of special interfaces to MC or
special user classes as well as some meta analysis packages. In the
example part of the book you will find how these interact to produce
completely reproducible analyses with just a view lines of code.

\section{hepstore.framework.herwig}

\subsection{Runcard}
\subsubsection{Collider}
\subsubsection{Beam}
\subsubsection{Model}
\subsubsection{Final State}
\subsubsection{Process}
\subsubsection{Cuts}
\subsubsection{Matrix Element}
\subsubsection{Scales}
\subsubsection{Shower}
\subsubsection{PDF}
\subsubsection{HepMC}
\subsubsection{Save}

\subsection{Generator}
generates a runcard from above,
build
integrate
find an empty run dir, take care of random seed
run MC
return folder where results are stored


\section{hepstore.framework.collider}


\section{hepstore.framework.eas}

The Extended Air Shower (EAS) framework provides a convienent tool box
to analyse the physics of cosmic rays. To our knowledge it is the only
tool box available, which allows for the simulation of any hard
physics interaction from any primary species.

\subsection{Cosmic Ray Interaction}
The interaction of the primary cosmic ray consists of two major
parts. First one needs to take care of the fact that it are actually
nuclei which collide. Second, we need to specify which sort of hard
interaction we are interested in.

\subsubsection{Nucleonic Modeling}
The simplest ansatz is to view the primary cosmic ray as an unbound
collection of neutrons and protons. One of these interacts with either
a proton or neutron from the air within earths athmosphere. The
remainder particles are viewed as secondary particles in the upcoming
air shower. Note, that this is what is actually used in hepstore-eas,
when one computes the hard interaction independently. Furthermore, the
opposite point of view would be a wounded nucleus which lost a proton
as secondary input for the air shower.

Of course, there are more sophisticated models. However, at the time
being, the only access from within hepstore is via direct simulation
with Corsika.

\subsubsection{Hard Interaction}
hepstore-eas uses the docker interface to simulate hard
interactions. At the time being, we provide access to the following MC
generators and physics processes:
%
\begin{description}
  \setstretch{1.5}
\item[Corsika] Di-jet production
\item[Herwig 7] Di-jet production; heavy resonance decaying to tops,
  jets, leptons, neutrinos, photons
\item[Herwig++] Sphalerons   
\end{description}
%

\subsection{Air Shower}
The air shower simulation is handled by Corsika which allows for
secondary particles to be showerd, too. The following modules compose
the Corsika installation interfaced in hepstore-eas:
%
\begin{description}
  \setstretch{1.5}
\item[Hadronic Interactions] ...
\end{description}
%

\subsection{Observables}
EAS are complicated objects containing millions of particles
populating the full $(x,p)$ phase space. Obviously, we have to reduce
complexity and use projected or integrated observables. In
hepstore-eas we include $\rho_\mu$ and $X_\text{max}$. The former is
the number of muons in between 500m and 600m of shower core
distance. The latter indicates the maximal shower penetration depth
into the atmosphere.


\subsection{Analysis}
Our analysis consists of two separated building blocks, following the
hepstore philosophy. The data itself is analysed with hepstore-school
to have outputs for comparison and cross validation.

\subsubsection{Machine Learning}
We use hepstore.core.school to learn from the observables
extracted. All provided learning algorithms may be used for comparison
here.

\subsubsection{Statistical Analysis}
Depending on the set of choosen hard interactions we can perform the
analyses contained in hepstore.core.statistic, e.g compute
significances or perform limit setting.
