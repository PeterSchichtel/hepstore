
The framework is the part of the code collecting the steering of
reproducible analyses. It consists of special interfaces to MC or
special user classes as well as some meta analysis packages. In the
example part of the book you will find how these interact to produce
completely reproducible analyses with just a view lines of code.

\section{hepstore.framework.monte\_carlo}
This section describes particular interfaces to Monte Carlo (MC)
tools. They are derived from hepstore.core.docker\_interface. To
provide simple and efficient access to reproducible analyses we
provide these specialized interfaces. All of them are connected to a
corresponding command line tool.

\subsection{hepstore.framework.monte\_carlo.herwig}
\subsection{hepstore.framework.monte\_carlo.sherpa}
\subsection{hepstore.framework.monte\_carlo.corsika}

\section{hepstore.framework.analysis}
This section collects meta analysis modules needed for certain kind of analyses.

\subsection{hepstore.framework.analysis.eas}

the hepstore-analysis module is a high level package to perform the
usual statistical computations needed in high energy physics. it
utilizes hepstore.school to provide signal and background
classification and extract statistical quantities such as the maximal
poissonian significance, the upper exclusion bound on the signal cross
section etc.

still missing: traditional cut analysis
