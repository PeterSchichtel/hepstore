%%%%%%%%%%%%%%%%%%%%%%%%%%%%%%%%%%%%%%%%%
% The Legrand Orange Book
% LaTeX Template
% Version 2.2 (30/3/17)
%
% This template has been downloaded from:
% http://www.LaTeXTemplates.com
%
% Original author:
% Mathias Legrand (legrand.mathias@gmail.com) with modifications by:
% Vel (vel@latextemplates.com)
%
% License:
% CC BY-NC-SA 3.0 (http://creativecommons.org/licenses/by-nc-sa/3.0/)
%
% Compiling this template:
% This template uses biber for its bibliography and makeindex for its index.
% When you first open the template, compile it from the command line with the 
% commands below to make sure your LaTeX distribution is configured correctly:
%
% 1) pdflatex main
% 2) makeindex main.idx -s StyleInd.ist
% 3) biber main
% 4) pdflatex main x 2
%
% After this, when you wish to update the bibliography/index use the appropriate
% command above and make sure to compile with pdflatex several times 
% afterwards to propagate your changes to the document.
%
% This template also uses a number of packages which may need to be
% updated to the newest versions for the template to compile. It is strongly
% recommended you update your LaTeX distribution if you have any
% compilation errors.
%
% Important note:
% Chapter heading images should have a 2:1 width:height ratio,
% e.g. 920px width and 460px height.
%
%%%%%%%%%%%%%%%%%%%%%%%%%%%%%%%%%%%%%%%%%

%----------------------------------------------------------------------------------------
%	PACKAGES AND OTHER DOCUMENT CONFIGURATIONS
%----------------------------------------------------------------------------------------

\documentclass[11pt,fleqn]{book} % Default font size and left-justified equations

%----------------------------------------------------------------------------------------

\input{structure} % Insert the commands.tex file which contains the majority of the structure behind the template

%----------------------------------------------------------------------------------------
%	USER INCLUDES
%----------------------------------------------------------------------------------------
\usepackage{framed}
\usepackage{setspace}
\usepackage{listings}
\usepackage{color}

\definecolor{dkgreen}{rgb}{0,0.6,0}
\definecolor{gray}{rgb}{0.5,0.5,0.5}
\definecolor{mauve}{rgb}{0.58,0,0.82}

\usepackage{inconsolata}

\definecolor{aqua}{rgb}{0.0, 1.0, 1.0}
\definecolor{babyblue}{rgb}{0.54, 0.81, 0.94}


\lstdefinelanguage{Python2}{%
  language     = Python,
  morekeywords = {self},
}


\lstset{frame=tb,
  language=Python2,
  aboveskip=3mm,
  belowskip=3mm,
  showstringspaces=false,
  columns=flexible,
  basicstyle={\small\ttfamily},
  numbers=none,
  numberstyle=\tiny\color{gray},
  keywordstyle=\color{blue},
  commentstyle=\color{dkgreen},
  stringstyle=\color{mauve},
  breaklines=true,
  breakatwhitespace=true,
  keepspaces=true
  tabsize=3,
  emph={ True, False, None},
  emphstyle={ \color{babyblue}}
}

\def\changemargin#1#2{\list{}{\rightmargin#2\leftmargin#1}\item[]}
\let\endchangemargin=\endlist

\usepackage{dirtree}
\usepackage{nccmath}

\usepackage{titlesec}
\titleformat*{\section}{\LARGE\bfseries}
\titleformat*{\subsection}{\Large\bfseries}
\titleformat*{\subsubsection}{\large\bfseries}
\titleformat*{\paragraph}{\small\bfseries}

\usepackage[useregional]{datetime2}

%----------------------------------------------------------------------------------------
%	USER DEFINITIONS
%----------------------------------------------------------------------------------------
\newcommand{\hepstore}{\texttt{HepStore}}
\newcommand{\python}{\texttt{Python~2.7}}
\newcommand{\docker}{\texttt{Docker}}

%----------------------------------------------------------------------------------------
%	BEGIN DOCUMENT
%----------------------------------------------------------------------------------------
\begin{document}

%----------------------------------------------------------------------------------------
%	TITLE PAGE
%----------------------------------------------------------------------------------------

\begingroup
\thispagestyle{empty}
\begin{tikzpicture}[remember picture,overlay]
\node[inner sep=0pt] (background) at (current page.center) {\includegraphics[width=\paperwidth]{background}};
\draw (current page.center) node [fill=ocre!30!white,fill opacity=0.6,text opacity=1,inner sep=1cm]{\Huge\centering\bfseries\sffamily\parbox[c][][t]{\paperwidth}{
    \begin{center}
      HepStore\\[15pt] % Book title
      {\Large -- The Awakening -- }\\[20pt] % Subtitle
      {\huge Peter Schichtel}
      %\\{\large Contributor(s):}
    \end{center}
    %{\small \hspace*{0.5cm}Previous authors(s):} \\
    %{\small \hspace*{0.5cm}Previous contributor(s): }
}}; % Author name
\end{tikzpicture}
\vfill
\endgroup

%----------------------------------------------------------------------------------------
%	COPYRIGHT PAGE
%----------------------------------------------------------------------------------------

\newpage
~\vfill
\thispagestyle{empty}

\noindent Copyright \copyright\ \the\year\ Peter Schichtel\\ % Copyright notice

\noindent \textsc{Published on ArXiv}\\ % Publisher

\noindent \textsc{https://github.com/PeterSchichtel}\\ % URL

\noindent Licensed under the Creative Commons Attribution-NonCommercial 3.0 Unported License (the ``License''). You may not use this file except in compliance with the License. You may obtain a copy of the License at \url{http://creativecommons.org/licenses/by-nc/3.0}. Unless required by applicable law or agreed to in writing, software distributed under the License is distributed on an \textsc{``as is'' basis, without warranties or conditions of any kind}, either express or implied. See the License for the specific language governing permissions and limitations under the License.\\ % License information

\noindent \textit{First printing \today} % Printing/edition date

%----------------------------------------------------------------------------------------
%	TABLE OF CONTENTS
%----------------------------------------------------------------------------------------

%\usechapterimagefalse % If you don't want to include a chapter image, use this to toggle images off - it can be enabled later with \usechapterimagetrue

\chapterimage{chapter_head_1.pdf} % Table of contents heading image

\pagestyle{empty} % No headers

\tableofcontents % Print the table of contents itself

\cleardoublepage % Forces the first chapter to start on an odd page so it's on the right

\pagestyle{fancy} % Print headers again

%----------------------------------------------------------------------------------------
%	PART 0
%----------------------------------------------------------------------------------------

\part{The Idea}

%----------------------------------------------------------------------------------------
%	CHAPTER 1
%----------------------------------------------------------------------------------------
\chapter{Installation}

\section{Command Line Installation}
One key idea of HepStore is simplicity. Therefore, HepStore is written
solely in Python. Rendering its installation as simple as:
%
\begin{framed}
  \begin{center}
    pip install  git+https://github.com/PeterSchichtel/hepstore.git
  \end{center}
\end{framed}
%
There are two optional arguments to the installation process:
%
\begin{enumerate}
\item -{}-upgrade ~~ in case the user likes to upgrade an older installation.
\item -{}-user ~~ in case the user whishes to perform a local user installation.
\end{enumerate}
%

\section{Prerequisites}
Even though simplicity and elegance are key points of HepStore's
development strategy. There are a few packages the code relies on.
%
\begin{description}
  \setstretch{1.5}
\item[Python 2.7] web adress
\item[Docker] web adress
\end{description}
%


%----------------------------------------------------------------------------------------
%	CHAPTER 2
%----------------------------------------------------------------------------------------
\chapter{Motivation}

\section{Reproducibility}
Reproducing any result from previous scientific work is at the heart
of the scientific method. It occures often that ground breaking
scientific results are doubted for many years until reproduced
correctly. Contrary, if not reproduced at all, the result is simply
not believed to be true~\cite{}. It is therefore obvious and vital to
any scientific effort to be easily reproduced. Of course, for some
experimental setup this just might not be easily done. It might be
that the experience needed to craft a certain item just has been
developed during the actual research itself. However, in the high
energy physics community, more and more scientific results completely
depend on work performed by computers. And indeed, there has been
tremondous effort in the experimantal LHC community to achive high
scientific standards of reproducibility. Namely the
\texttt{REANA}~\cite{} frame work.

However, one should not think that no such efforts have been
undertaken in the phenomelogical community. There is
\texttt{MadAnalysis5}~\cite{} which has the ability to recast LHC
analyses and provides a convienent frame work for high energy physics
analyses. There is, of course, \texttt{Rivet}~\cite{}, which comes
along with \emph{actual, verified} phenomenolgical and experimental
analyses. There is the \texttt{ROOT}~\cite{} framework which provides
a multitude of analysis tools. These tools provide building blocks to
perform, save and eventually redo any analysis. In addition, the Monte
Carlo (MC) code developers~\cite{} provide ever simpler installation
routines for their code packages, including interfaces to third party
software.

So, why bother with the reproducibility of phenomenoligical results?
It is true that a fair amount of the analysis itself is either
automated or can be easily build from freely availbale
software. However, that does not mean that it is actually
reproducible. Anybody who has ever tried to redo what they or someone
else did five years ago might already know some of the problems:
library missing, code missing, data missing, does not compile anymore,
does not run anymore, 32-bit what?, not supported, what was the random
seed again, which version of package XY was used, etc. pp. Let us
start with defining the neccessary conditions for a phenomenolgical
analysis to be \emph{fully reproducible}.
%
\begin{enumerate}
\item The state of the machine can be {\bf frozen} and published.
\item Except for the state of the machine, the work only depends on
  simple (i.e text), publishable {\bf input files}.
\item The full analysis chain can be invoked by means of a
  {\bf platform independent} command.
\end{enumerate}
%

Requirement (a) is most easily solved by invoking
\docker~\cite{}. \docker~is a framework which allows to ship
containerized applications. This means it is able to produce a frozen
state of a machine. The \docker~image contains all the libraries and
packages installed during its creation. Furthermore, \docker~hosts a
webpage where one can upload and therefore publish \docker~images.

Requirement (b) is most of the time already satisfied by todays
publications, where run and parameter cards are added as supplimental
material.

Requirement (c) is more tricky, though. Furthermore, it needs
motivation as its neccassity is not as straight forward as (a) and
(b). Note, that the aim of the above description is the full
reproduction of a phenomenoligical analysis. This includes not only
the state of the machine and any input parameters, but also what
commands to run in which order to perform the full analysis. This also
includes any code written specifically for this analysis. Furthermore,
we require platform independence. Who ever replicates the analysis
should not fail due to any special platform requirements.


\section{Hep Phenomenology from automated Building Blocks}

\subsection{Analysis and Data Types}

\subsection{Machine Learning}

\subsection{Statistics}


\section{The Community}


%----------------------------------------------------------------------------------------
%	CHAPTER 3
%----------------------------------------------------------------------------------------
\chapter{Market Place}
\input{part0/market.tex}




%----------------------------------------------------------------------------------------
%	PART I
%----------------------------------------------------------------------------------------

\part{The Code}

%----------------------------------------------------------------------------------------
%	CHAPTER 1
%----------------------------------------------------------------------------------------
\chapter{Installation}

\section{Command Line Installation}
One key idea of HepStore is simplicity. Therefore, HepStore is written
solely in Python. Rendering its installation as simple as:
%
\begin{framed}
  \begin{center}
    pip install  git+https://github.com/PeterSchichtel/hepstore.git
  \end{center}
\end{framed}
%
There are two optional arguments to the installation process:
%
\begin{enumerate}
\item -{}-upgrade ~~ in case the user likes to upgrade an older installation.
\item -{}-user ~~ in case the user whishes to perform a local user installation.
\end{enumerate}
%

\section{Prerequisites}
Even though simplicity and elegance are key points of HepStore's
development strategy. There are a few packages the code relies on.
%
\begin{description}
  \setstretch{1.5}
\item[Python 2.7] web adress
\item[Docker] web adress
\end{description}
%


%----------------------------------------------------------------------------------------
%	CHAPTER 2
%----------------------------------------------------------------------------------------
\chapter{Motivation}

\section{Reproducibility}
Reproducing any result from previous scientific work is at the heart
of the scientific method. It occures often that ground breaking
scientific results are doubted for many years until reproduced
correctly. Contrary, if not reproduced at all, the result is simply
not believed to be true~\cite{}. It is therefore obvious and vital to
any scientific effort to be easily reproduced. Of course, for some
experimental setup this just might not be easily done. It might be
that the experience needed to craft a certain item just has been
developed during the actual research itself. However, in the high
energy physics community, more and more scientific results completely
depend on work performed by computers. And indeed, there has been
tremondous effort in the experimantal LHC community to achive high
scientific standards of reproducibility. Namely the
\texttt{REANA}~\cite{} frame work.

However, one should not think that no such efforts have been
undertaken in the phenomelogical community. There is
\texttt{MadAnalysis5}~\cite{} which has the ability to recast LHC
analyses and provides a convienent frame work for high energy physics
analyses. There is, of course, \texttt{Rivet}~\cite{}, which comes
along with \emph{actual, verified} phenomenolgical and experimental
analyses. There is the \texttt{ROOT}~\cite{} framework which provides
a multitude of analysis tools. These tools provide building blocks to
perform, save and eventually redo any analysis. In addition, the Monte
Carlo (MC) code developers~\cite{} provide ever simpler installation
routines for their code packages, including interfaces to third party
software.

So, why bother with the reproducibility of phenomenoligical results?
It is true that a fair amount of the analysis itself is either
automated or can be easily build from freely availbale
software. However, that does not mean that it is actually
reproducible. Anybody who has ever tried to redo what they or someone
else did five years ago might already know some of the problems:
library missing, code missing, data missing, does not compile anymore,
does not run anymore, 32-bit what?, not supported, what was the random
seed again, which version of package XY was used, etc. pp. Let us
start with defining the neccessary conditions for a phenomenolgical
analysis to be \emph{fully reproducible}.
%
\begin{enumerate}
\item The state of the machine can be {\bf frozen} and published.
\item Except for the state of the machine, the work only depends on
  simple (i.e text), publishable {\bf input files}.
\item The full analysis chain can be invoked by means of a
  {\bf platform independent} command.
\end{enumerate}
%

Requirement (a) is most easily solved by invoking
\docker~\cite{}. \docker~is a framework which allows to ship
containerized applications. This means it is able to produce a frozen
state of a machine. The \docker~image contains all the libraries and
packages installed during its creation. Furthermore, \docker~hosts a
webpage where one can upload and therefore publish \docker~images.

Requirement (b) is most of the time already satisfied by todays
publications, where run and parameter cards are added as supplimental
material.

Requirement (c) is more tricky, though. Furthermore, it needs
motivation as its neccassity is not as straight forward as (a) and
(b). Note, that the aim of the above description is the full
reproduction of a phenomenoligical analysis. This includes not only
the state of the machine and any input parameters, but also what
commands to run in which order to perform the full analysis. This also
includes any code written specifically for this analysis. Furthermore,
we require platform independence. Who ever replicates the analysis
should not fail due to any special platform requirements.


\section{Hep Phenomenology from automated Building Blocks}

\subsection{Analysis and Data Types}

\subsection{Machine Learning}

\subsection{Statistics}


\section{The Community}


%----------------------------------------------------------------------------------------
%	CHAPTER 3
%----------------------------------------------------------------------------------------
\chapter{Market Place}
\input{part0/market.tex}




%----------------------------------------------------------------------------------------
%	PART II
%----------------------------------------------------------------------------------------

\part{Examples}

%----------------------------------------------------------------------------------------
%	CHAPTER 1
%----------------------------------------------------------------------------------------
\chapter{Installation}

\section{Command Line Installation}
One key idea of HepStore is simplicity. Therefore, HepStore is written
solely in Python. Rendering its installation as simple as:
%
\begin{framed}
  \begin{center}
    pip install  git+https://github.com/PeterSchichtel/hepstore.git
  \end{center}
\end{framed}
%
There are two optional arguments to the installation process:
%
\begin{enumerate}
\item -{}-upgrade ~~ in case the user likes to upgrade an older installation.
\item -{}-user ~~ in case the user whishes to perform a local user installation.
\end{enumerate}
%

\section{Prerequisites}
Even though simplicity and elegance are key points of HepStore's
development strategy. There are a few packages the code relies on.
%
\begin{description}
  \setstretch{1.5}
\item[Python 2.7] web adress
\item[Docker] web adress
\end{description}
%


%----------------------------------------------------------------------------------------
%	CHAPTER 2
%----------------------------------------------------------------------------------------
\chapter{Motivation}

\section{Reproducibility}
Reproducing any result from previous scientific work is at the heart
of the scientific method. It occures often that ground breaking
scientific results are doubted for many years until reproduced
correctly. Contrary, if not reproduced at all, the result is simply
not believed to be true~\cite{}. It is therefore obvious and vital to
any scientific effort to be easily reproduced. Of course, for some
experimental setup this just might not be easily done. It might be
that the experience needed to craft a certain item just has been
developed during the actual research itself. However, in the high
energy physics community, more and more scientific results completely
depend on work performed by computers. And indeed, there has been
tremondous effort in the experimantal LHC community to achive high
scientific standards of reproducibility. Namely the
\texttt{REANA}~\cite{} frame work.

However, one should not think that no such efforts have been
undertaken in the phenomelogical community. There is
\texttt{MadAnalysis5}~\cite{} which has the ability to recast LHC
analyses and provides a convienent frame work for high energy physics
analyses. There is, of course, \texttt{Rivet}~\cite{}, which comes
along with \emph{actual, verified} phenomenolgical and experimental
analyses. There is the \texttt{ROOT}~\cite{} framework which provides
a multitude of analysis tools. These tools provide building blocks to
perform, save and eventually redo any analysis. In addition, the Monte
Carlo (MC) code developers~\cite{} provide ever simpler installation
routines for their code packages, including interfaces to third party
software.

So, why bother with the reproducibility of phenomenoligical results?
It is true that a fair amount of the analysis itself is either
automated or can be easily build from freely availbale
software. However, that does not mean that it is actually
reproducible. Anybody who has ever tried to redo what they or someone
else did five years ago might already know some of the problems:
library missing, code missing, data missing, does not compile anymore,
does not run anymore, 32-bit what?, not supported, what was the random
seed again, which version of package XY was used, etc. pp. Let us
start with defining the neccessary conditions for a phenomenolgical
analysis to be \emph{fully reproducible}.
%
\begin{enumerate}
\item The state of the machine can be {\bf frozen} and published.
\item Except for the state of the machine, the work only depends on
  simple (i.e text), publishable {\bf input files}.
\item The full analysis chain can be invoked by means of a
  {\bf platform independent} command.
\end{enumerate}
%

Requirement (a) is most easily solved by invoking
\docker~\cite{}. \docker~is a framework which allows to ship
containerized applications. This means it is able to produce a frozen
state of a machine. The \docker~image contains all the libraries and
packages installed during its creation. Furthermore, \docker~hosts a
webpage where one can upload and therefore publish \docker~images.

Requirement (b) is most of the time already satisfied by todays
publications, where run and parameter cards are added as supplimental
material.

Requirement (c) is more tricky, though. Furthermore, it needs
motivation as its neccassity is not as straight forward as (a) and
(b). Note, that the aim of the above description is the full
reproduction of a phenomenoligical analysis. This includes not only
the state of the machine and any input parameters, but also what
commands to run in which order to perform the full analysis. This also
includes any code written specifically for this analysis. Furthermore,
we require platform independence. Who ever replicates the analysis
should not fail due to any special platform requirements.


\section{Hep Phenomenology from automated Building Blocks}

\subsection{Analysis and Data Types}

\subsection{Machine Learning}

\subsection{Statistics}


\section{The Community}


%----------------------------------------------------------------------------------------
%	CHAPTER 3
%----------------------------------------------------------------------------------------
\chapter{Market Place}
\input{part0/market.tex}




%----------------------------------------------------------------------------------------
%	PART III
%----------------------------------------------------------------------------------------

\part{Sortie de Scene}

%----------------------------------------------------------------------------------------
%	CHAPTER 1
%----------------------------------------------------------------------------------------
\chapter{Installation}

\section{Command Line Installation}
One key idea of HepStore is simplicity. Therefore, HepStore is written
solely in Python. Rendering its installation as simple as:
%
\begin{framed}
  \begin{center}
    pip install  git+https://github.com/PeterSchichtel/hepstore.git
  \end{center}
\end{framed}
%
There are two optional arguments to the installation process:
%
\begin{enumerate}
\item -{}-upgrade ~~ in case the user likes to upgrade an older installation.
\item -{}-user ~~ in case the user whishes to perform a local user installation.
\end{enumerate}
%

\section{Prerequisites}
Even though simplicity and elegance are key points of HepStore's
development strategy. There are a few packages the code relies on.
%
\begin{description}
  \setstretch{1.5}
\item[Python 2.7] web adress
\item[Docker] web adress
\end{description}
%


%----------------------------------------------------------------------------------------
%	CHAPTER 2
%----------------------------------------------------------------------------------------
\chapter{Motivation}

\section{Reproducibility}
Reproducing any result from previous scientific work is at the heart
of the scientific method. It occures often that ground breaking
scientific results are doubted for many years until reproduced
correctly. Contrary, if not reproduced at all, the result is simply
not believed to be true~\cite{}. It is therefore obvious and vital to
any scientific effort to be easily reproduced. Of course, for some
experimental setup this just might not be easily done. It might be
that the experience needed to craft a certain item just has been
developed during the actual research itself. However, in the high
energy physics community, more and more scientific results completely
depend on work performed by computers. And indeed, there has been
tremondous effort in the experimantal LHC community to achive high
scientific standards of reproducibility. Namely the
\texttt{REANA}~\cite{} frame work.

However, one should not think that no such efforts have been
undertaken in the phenomelogical community. There is
\texttt{MadAnalysis5}~\cite{} which has the ability to recast LHC
analyses and provides a convienent frame work for high energy physics
analyses. There is, of course, \texttt{Rivet}~\cite{}, which comes
along with \emph{actual, verified} phenomenolgical and experimental
analyses. There is the \texttt{ROOT}~\cite{} framework which provides
a multitude of analysis tools. These tools provide building blocks to
perform, save and eventually redo any analysis. In addition, the Monte
Carlo (MC) code developers~\cite{} provide ever simpler installation
routines for their code packages, including interfaces to third party
software.

So, why bother with the reproducibility of phenomenoligical results?
It is true that a fair amount of the analysis itself is either
automated or can be easily build from freely availbale
software. However, that does not mean that it is actually
reproducible. Anybody who has ever tried to redo what they or someone
else did five years ago might already know some of the problems:
library missing, code missing, data missing, does not compile anymore,
does not run anymore, 32-bit what?, not supported, what was the random
seed again, which version of package XY was used, etc. pp. Let us
start with defining the neccessary conditions for a phenomenolgical
analysis to be \emph{fully reproducible}.
%
\begin{enumerate}
\item The state of the machine can be {\bf frozen} and published.
\item Except for the state of the machine, the work only depends on
  simple (i.e text), publishable {\bf input files}.
\item The full analysis chain can be invoked by means of a
  {\bf platform independent} command.
\end{enumerate}
%

Requirement (a) is most easily solved by invoking
\docker~\cite{}. \docker~is a framework which allows to ship
containerized applications. This means it is able to produce a frozen
state of a machine. The \docker~image contains all the libraries and
packages installed during its creation. Furthermore, \docker~hosts a
webpage where one can upload and therefore publish \docker~images.

Requirement (b) is most of the time already satisfied by todays
publications, where run and parameter cards are added as supplimental
material.

Requirement (c) is more tricky, though. Furthermore, it needs
motivation as its neccassity is not as straight forward as (a) and
(b). Note, that the aim of the above description is the full
reproduction of a phenomenoligical analysis. This includes not only
the state of the machine and any input parameters, but also what
commands to run in which order to perform the full analysis. This also
includes any code written specifically for this analysis. Furthermore,
we require platform independence. Who ever replicates the analysis
should not fail due to any special platform requirements.


\section{Hep Phenomenology from automated Building Blocks}

\subsection{Analysis and Data Types}

\subsection{Machine Learning}

\subsection{Statistics}


\section{The Community}


%----------------------------------------------------------------------------------------
%	CHAPTER 3
%----------------------------------------------------------------------------------------
\chapter{Market Place}
\input{part0/market.tex}






%----------------------------------------------------------------------------------------
%	BIBLIOGRAPHY
%----------------------------------------------------------------------------------------

\chapter*{Bibliography}
\addcontentsline{toc}{chapter}{\textcolor{ocre}{Bibliography}}
\section*{Books}
\addcontentsline{toc}{section}{Books}
\printbibliography[heading=bibempty,type=book]
\section*{Articles}
\addcontentsline{toc}{section}{Articles}
\printbibliography[heading=bibempty,type=article]

%----------------------------------------------------------------------------------------
%	INDEX
%----------------------------------------------------------------------------------------

\cleardoublepage
\phantomsection
\setlength{\columnsep}{0.75cm}
\addcontentsline{toc}{chapter}{\textcolor{ocre}{Index}}
\printindex

%----------------------------------------------------------------------------------------

\end{document}
