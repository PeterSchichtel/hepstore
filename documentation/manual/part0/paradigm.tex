
\section{Building Blockes}

\subsection{Atomic}
The first corner stone for the hepstore code is the idea of atomic
programming. Simply, try to find the smallest pieces possible to do
something and compose more complex structures from these building
blocks.

\subsection{Completeness}
The code should be completely self contained, in the sense that there
is either native hepstore implementation or an interface fitting into
the hepstore workflow.

\subsection{Efficiency and Elegance}
Besides the fact that the code should be modular, i.e. atomic, another
aim is efficience and elegance. What is the shortes way to make
something work in a certain way? How can something be composed using
the least building block or commands neccessary?

\subsection{Future Save}
As reproducibility is one of the key motivations of the project. At
the time being everything is written in Python 2.7. Python being a
platform independent tool should provide the possibility to use
hepstore on any machine in the future.

\subsection{Data Structure}
There are many custom data formats, especially in hep
phenomenology. However, to comply with the points raised above,
hepstore internally and externally uses numpy arrays, which may be
stored in the '.npy' format.

\section{Community Driven}
Ultimately the aims defined in Chapter~\ref{chap:motivation} only can
be reached by a community effort. The code structure needs to be
versioned in a public repository, where evrybody can in principle
contribute. The hepstore repository should contain the full
information.
%
\begin{framed}
  \begin{center}
    https://github.com/PeterSchichtel/hepstore.git
  \end{center}
\end{framed}
%
Scattering to different places is not wished for. Therefore, not only
the source code, but also the manual and the example files are
contained in the git repository. Furthermore, the code should be
structured in a way helping to easily identify where to place a
certain contribution.

\subsection{Code}
%
\begin{description}
\item[Core] The core is defined by all genuine hepstore modules and
  classes.\\
\item[Framework] On the other hand the framework is defined to contain
  the actual user interfaces, e.g the framework of objects to work
  with. Furthermore, the meta steering for the reproduction of
  analyses lives here.
\end{description}
%

\subsection{Documentation}
%
\begin{description}
\item[Examples] To ease the usage of hepstore extensive examples are
  given. Contributors should add examples demonstrating the usage of
  their additions.\\
\item[Manual] The last pillar is the written documentation. Any change
  in code should reflect in a change in the manual. Naturally, the
  manual consists of two distinct parts. One concerned with the code
  itself, the other with the examples.
\end{description}
%
