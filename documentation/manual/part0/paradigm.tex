
\section{Atomic}
The first corner stone for the hepstore code is the idea of atomic
programming. Simply, try to find the smallest pieces possible to do
something and compose more complex structures from these building
blocks.

\section{Efficiency and Elegance}
Besides the fact that the code should be modular, i.e. atomic, another
aim is efficience and elegance. What is the shortes way to make
something work in a certain way? How can something be composed using
the least building block or commands neccessary?

\section{Future Save}
As reproducibility is one of the key motivations of the project. At
the time being everything is written in Python 2.7. Python being a
platform independent tool should provide the possibility to use
hepstore on any machine in the future.

\section{Community Driven}
Ultimately the aims defined in Chapter~\ref{chap:motivation} only can
be reached by a community effort. The code structure needs to be
versioned in a public repository, where evrybody can in principle
contribute. Furthermore, the code should be structured in a way
helping to easily identify where to place a certain contribution.

\section{Places}
The hepstore repository should contain the full
information. Scattering to different places is not wished
for. Therefore, not only the source code, but also the manual and the
esample files are contained in the git repository.
